\section{Nociones de la memoria del computador}
\subsection{Preguntas}
\begin{itemize}
    \item Defina que es la memoria del computador
    \item Menciones los tipos de memoria que conoce y haga una pequeña descripción de cada tipo
    \item Describa la manera como se gestiona la memoria del computador
    \item ¿Qué hace que una memoria sea mas rápida que otra? ¿Por qué esto es importante?
\end{itemize}

\newpage

\section{Solución}
\begin{itemize}

    \subsection{Defina que es la memoria del computador}
    \item Cumpliendo realmente el papel más importante en cualquier dispositivo electrónico, existen diversos dispositivos de almacenamiento de información, su función en esencia es la manipulación de datos (almacenamiento, lectura y escritura).
    
    Cada tipo de memoria que se encuentra en nuestras computadoras cumple una función muy importante, las características de cada una de ellas depende de su uso y labor en el proceso de manipulación de datos. Desde pequeñas en almacenamiento pero con gran velocidad, hasta memorias con gran capacidad de almacenamiento pero de velocidad más reducida.
    
    Concluyendo la idea de lo que es la memoria del computador y conociendo un poco características de estas, definimos a la memoria como un dispositivo donde almacenamos información con la que trabajan otros componentes de la computadora para ser procesada y retornar valores deseados de lectura y escritura.
    
    
    
    \subsection{ Menciones los tipos de memoria que conoce y haga una pequeña descripción de cada tipo}
    \item \textbf{Memoria ROM: } Dispositivo solamente de lectura, sus datos son gravados a penas una una vez en este tipo de memoria, y una de sus principales ventajas es su capacidad de almacenamiento y almacenamiento de datos de manera no volátil, es decir, sus datos no son perdidos en ausencia de energía eléctrica.
    
    Posee 6 subcategorías:
    
    PROM Programable, solo lectura.
    EPROM Capacidad de permitir que datos sean regrabados en el dispositivo.
    EEPROM
    EAROM
    FLASH.
    CD-ROM / DVD-ROM y similares.
    
    
    \subsection{ Describa la manera como se gestiona la memoria del computador}
    \item -----------
    
    
    \subsection{ ¿Qué hace que una memoria sea mas rápida que otra? ¿Por qué esto es importante?}
    \item ------------
    
    
\end{itemize}