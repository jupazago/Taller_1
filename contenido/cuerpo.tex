\section{Nociones de la memoria del computador}
\subsection{Preguntas}
\begin{itemize}
    \item Defina que es la memoria del computador
    \item Menciones los tipos de memoria que conoce y haga una pequeña descripción de cada tipo
    \item Describa la manera como se gestiona la memoria del computador
    \item ¿Qué hace que una memoria sea mas rápida que otra? ¿Por qué esto es importante?
\end{itemize}

\newpage

\section{Solución}
\begin{itemize}

    \subsection{Defina que es la memoria del computador}
    \item Cumpliendo realmente el papel más importante en cualquier dispositivo electrónico, existen diversos dispositivos de almacenamiento de información, su función en esencia es la manipulación de datos (almacenamiento, lectura y escritura).
    
    Cada tipo de memoria que se encuentra en nuestras computadoras cumple una función muy importante, las características de cada una de ellas depende de su uso y labor en el proceso de manipulación de datos. Desde pequeñas en almacenamiento pero con gran velocidad, hasta memorias con gran capacidad de almacenamiento pero de velocidad más reducida.
    
    Concluyendo la idea de lo que es la memoria del computador y conociendo un poco características de estas, definimos a la memoria como un dispositivo donde almacenamos información con la que trabajan otros componentes de la computadora para ser procesada y retornar valores deseados de lectura y escritura.
    
    
    
    \subsection{ Menciones los tipos de memoria que conoce y haga una pequeña descripción de cada tipo}
    \item \textbf{Memoria ROM: } Dispositivo solamente de lectura, sus datos son gravados a penas una una vez en este tipo de memoria, y una de sus principales ventajas es su capacidad de almacenamiento y almacenamiento de datos de manera no volátil, es decir, sus datos no son perdidos en ausencia de energía eléctrica.
    
    Posee 6 subcategorías:
    
    \textbf{PROM:} Programable, solo lectura. \newline
    \textbf{EPROM:} Capacidad de permitir que datos sean regrabados en el dispositivo. \newline
    \textbf{EEPROM:} Este tipo de memoria también permite el almacenamiento de datos reprogramada eléctricamente. \newline
    \textbf{EAROM:} De lectura alterable eléctricamente. \newline
    \textbf{FLASH:} permite la lectura y escritura de múltiples posiciones de memoria en la misma operación \newline
    \textbf{CD-ROM / DVD-ROM y similares} Disco compacto que usa rayos láser para leer información. \newline
    
    \item \textbf{Memoria RAM: } Es una de las partes mas importantes del computador, en ella el procesador almacena los datos temporalmente de manera volátil, se encarga de llevar información y datos desde la memoria ROM al procesador donde luego se ejecutaran diversos procesos.
    
    Posee 3 subcategorías:
    
    \textbf{SRAM:} Está basada en semiconductores, memoria estática. \newline
    \textbf{DRAM:} Está basada en condensadores donde pierden cargan progresivamente y necesitan de un circuito de refresco, memoria dinámica. \newline
    \textbf{MRAM:} Este tipo de memoria se basa en magnetismo, su manera de gestionar los datos es igual a las anteriores. \newline
    
    
    \subsection{ Describa la manera como se gestiona la memoria del computador}
    \item -----------
    
    
    \subsection{ ¿Qué hace que una memoria sea mas rápida que otra? ¿Por qué esto es importante?}
    \item ------------
    
    
\end{itemize}