\section{Nociones de la memoria del computador}
\subsection{Preguntas}
\begin{itemize}
    \item Defina que es la memoria del computador
    \item Menciones los tipos de memoria que conoce y haga una pequeña descripción de cada tipo
    \item Describa la manera como se gestiona la memoria del computador
    \item ¿Qué hace que una memoria sea mas rápida que otra? ¿Por qué esto es importante?
\end{itemize}

\newpage

\section{Solución}
\begin{itemize}

    \subsection{Defina que es la memoria del computador}
    \item Cumpliendo realmente el papel más importante en cualquier dispositivo electrónico, existen diversos dispositivos de almacenamiento de información, su función en esencia es la manipulación de datos (almacenamiento, lectura y escritura).
    
    Cada tipo de memoria que se encuentra en nuestras computadoras cumple una función muy importante, las características de cada una de ellas depende de su uso y labor en el proceso de manipulación de datos. Desde pequeñas en almacenamiento pero con gran velocidad, hasta memorias con gran capacidad de almacenamiento pero de velocidad más reducida.
    
    Concluyendo la idea de lo que es la memoria del computador y conociendo un poco características de estas, definimos a la memoria como un dispositivo donde almacenamos información con la que trabajan otros componentes de la computadora para ser procesada y retornar valores deseados de lectura y escritura.
    
    
    
    \subsection{ Menciones los tipos de memoria que conoce y haga una pequeña descripción de cada tipo}
    \item \textbf{Memoria ROM: } Dispositivo solamente de lectura, sus datos son gravados a penas una una vez en este tipo de memoria, y una de sus principales ventajas es su capacidad de almacenamiento y almacenamiento de datos de manera no volátil, es decir, sus datos no son perdidos en ausencia de energía eléctrica.
    
    Posee 6 subcategorías:
    
    \textbf{PROM:} Programable, solo lectura. \newline
    \textbf{EPROM:} Capacidad de permitir que datos sean regrabados en el dispositivo. \newline
    \textbf{EEPROM:} Este tipo de memoria también permite el almacenamiento de datos reprogramada eléctricamente. \newline
    \textbf{EAROM:} De lectura alterable eléctricamente. \newline
    \textbf{FLASH:} permite la lectura y escritura de múltiples posiciones de memoria en la misma operación \newline
    \textbf{CD-ROM / DVD-ROM y similares} Disco compacto que usa rayos láser para leer información. \newline
    
    \item \textbf{Memoria RAM: } Es una de las partes mas importantes del computador, en ella el procesador almacena los datos temporalmente de manera volátil, se encarga de llevar información y datos desde la memoria ROM al procesador donde luego se ejecutaran diversos procesos.
    
    Posee 3 subcategorías:
    
    \textbf{SRAM:} Está basada en semiconductores, memoria estática. \newline
    \textbf{DRAM:} Está basada en condensadores donde pierden cargan progresivamente y necesitan de un circuito de refresco, memoria dinámica. \newline
    \textbf{MRAM:} Este tipo de memoria se basa en magnetismo, su manera de gestionar los datos es igual a las anteriores. \newline
    
    \item \textbf{Memoria Cache: } Esta área de almacenamiento está dedicada a los datos usados o solicitados con más frecuencia para su recuperación a gran velocidad.
    
    Posee 3 subcategorías:
    
    \textbf{L1:} Es la que posee mayor velocidad y menor capacidad \newline
    \textbf{L2:} Más lenta que la anterior pero posee una mayor capacidad de almacenamiento de información \newline
    \textbf{L3:} Es la que posee mayor capacidad de almacenamiento pero a su vez es la más lenta, aunque sigue siendo mas veloz que la memoria RAM. \newline
    
    
    \subsection{ Describa la manera como se gestiona la memoria del computador}
    \item La gestión de memoria inicia desde el primer momento en el que se enciende la computadora, existe una memoria no volátil que se encuentra en la placa madre y se encarga de dar instrucciones para el análisis de los componentes de la computadora para comunicar si existen fallos, siendo todo correcto se efectúa el arranque normal del dispositivo, ya en funcionamiento la BIOS se encarga provee información del sistema operativo almacenada en la ROM, la RAM se encarga de traer dicha información y llevarla al procesador donde se ejecuta el arranque del sistema, mismo sistema que estará siendo utilizado en todo momento hasta el apagado de la computadora.
    
    Una vez cargado completamente el sistema podemos ejecutar un sin fin de aplicaciones que serán cargadas a la memoria RAM trayendo información del disco duro, principalmente solo carga las partes mas primordiales de la aplicación y posteriormente se ejecutarán los otros fragmentos según sea solicitado.
    
    La memoria RAM se encarga en todo momento de acceder al disco duro (SSD ó HDD), lee la información y la lleva al procesador, se ejecutan procesos y donde a medida que no se necesite información se limpia y así genera nuevos espacios de memoria que anteriormente estaban siendo ocupados. Al cerrar una apelación toda su información también es eliminada de la memoria RAM.
    
    
    \subsection{ ¿Qué hace que una memoria sea mas rápida que otra? ¿Por qué esto es importante?}
    \item ------------
    
    
\end{itemize}